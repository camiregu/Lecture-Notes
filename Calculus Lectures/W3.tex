% camiregu 2024-jan-23
\chapter{(no title yet)}

%----------------------------------------------------------------------------------------

\begin{corollary}
    With the assumptions and notation of the IPFT, let $S = \set{(x,y) \in \Omega: F(x,y) = c}.$ Then $S \cap (U \times V) = \set{(x,y) \in \R^{n + m}: x \in U \text{ and } y = f\of{x}}$.
\end{corollary}

\begin{remark}
    Note that when $m=1$, then $\det\of{\frac{\partial F}{\partial y}} = \frac{\partial F}{\partial y}$. So if $\frac{\partial F}{\partial y}\of{a,b} \neq 0$ then the level set $S = \set{(x,y) \in \R^{n+1}: F\of{x,y} = c}$ in a neighbourhood of $(a,b)$ is the graph of the implicit function.
\end{remark}

\begin{example}(IPFT, level set, and graph)
    Consider the level set $S = \set{(x,y) \in \R^2: x^3y^2+y^3(x-1)^2 = 1}$ of $F\of{x,y} = x^3y^2 + y^3(x-1)^2$.
    \begin{enumerate}
        \item Show that $S$ is not the graph of any function $y = f(x)$, i.e., $S \neq \set{(x,y) \in \R^2: y = f\of{x}}$.
        \item Show that in a neighbourhood of $(1,1)$, $S$ is the graph of a smooth function $f$ and find the slope of the tangent line to the graph of $f$ at $(1,1)$.
    \end{enumerate}

    Solutions:
    \begin{enumerate}
        \item $(1,-1), (1,1) \in S$, so no such function exists.
        \item $\eval{\frac{\partial F}{\partial y}\of{1,1} = 2x^3y + 3^2(x-1)^2}_{x=1, y=1} = 2 \neq 0$. So by the IPFT (with $a=b=c=1$) and the corollary there exist open sets $U, V \in \R$ with $(1,1) \in U \times V$ and a smooth function $f: U \to V$ such that $f\of{1} = 1$, $F\of{x, f\of{x}} = 1 = x^3f\of{x}^2 + f\of{x}^3(x-1)^2 = 1$ for all $x \in U$, and $S \cap (U \times V) = \set{(x,y): x \in U \text{ and } y = f\of{x}}$.
        
        The slope is $f\inv\of{1}$: Since $x^3 f\of{x}^2 + f\of{x}^3 (x-1)^2 = 1$ for all $x \in U$, so $0 = \frac{\mathrm{d}}{\mathrm{d}x}\left[x^3 f\of{x}^2 + f\of{x}^3 (x-1)^2\right] = 3x^2 f\of{x}^2 + 2x^3 f\of{x} f'\of{x} + 3f\of{x}^2 f'\of{x} (x-1)^2 + 2f\of{x}^3 (x-1)$.
        When $x = 1$,  $f(1) = 1$, and so $0 = 3 + 2f'(1)$. Thus $f'(1) = \frac{3}{2}$
    \end{enumerate}
\end{example}

\begin{example}(Finding the derivative without the function)
    Consider the problem of solving the system of equations: $\left\{ \begin{matrix} xy^2 + xzu + yv^2 = 3 \\ u^3yz + 2xv - u^2v^2 = 2 \end{matrix} \right.$ (*).
    for $u$ and $v$ in terms of $x, y, z$ near $x = y = z = u, v= 1$ and computing the partial $\frac{\partial u}{\partial z}, \frac{\partial v}{\partial z}$.

    Let $a = (1,1,1), b = (1,1), c = (3,2),$ and $F: \R^3 \to \R^2 \to \R^2$ be given by 
    \[F(x,y,z,u,v) = (xy^2 + xzu + yu^2, u^3yz + 2xv - u^2v^2).\]
    Then $F(a,b) = c$, $\frac{\partial F}{\partial(u,v)} = \begin{bmatrix}
        xz & 2yv \\ 3u^2yz - 2uv^2 & 2x - 2u^2v
    \end{bmatrix}$.
    \[\det\of{\frac{\partial F}{\partial (u,v)}\of{a,b}} =\det \begin{bmatrix} 1 & 2 \\ 1 & 0 \end{bmatrix} = -2 \neq 0.\]

    Hence, by the IPFT there exists a smooth function $f(x,y,z) = (f_1\of{x,y,z}, f_2\of{x,y,z})$ defined on a neighbourhood $U$ of $u = (1,1,1)$ such that $F\of{x,y,z, f\of{x,y,z}} = (3,2) = c$ for all $(x,y,z) \in U$ and $f\of{1,1,1} = (1,1)$: $u = f_1\of{x,y,z}, v = f_2\of{x,y,z}$ are the expressions of $u$ and $v$ in terms of $x,y,z$. To find $\frac{\partial u}{\partial z}$ and $\frac{\partial v}{\partial z}$ we differentiate Eqs(*) with respect to $z$, treating $u$ and $v$ as functions of $x,y,z$:
    \begin{align*}
        0 &= \frac{\partial}{\partial z}\of{xy^2 + xzu + yv^2} = xu + xz \frac{\partial u}{\partial z} + 2yv \frac{\partial v}{\partial z} \\
        0 &= \frac{\partial}{\partial z}\of{u^3yz + 2xv - u^2v^2} = 3u^2 \frac{\partial u}{\partial z} yz + u^3y + 2x \frac{\partial v}{\partial z} - 2u \frac{\partial u}{\partial z}v^2 - u^2 2v \frac{\partial v}{\partial z}
    \end{align*}
    With $(x,y,z) = (1,1,1)$, $(u,v) = (1,1)$ we get
    \[1 + \frac{\partial u}{\partial z} + 2 \frac{\partial v}{\partial z} = 0, \frac{\partial u}{\partial z} + 1 = 0.\]

    Hence, $\frac{\partial f_1}{\partial z} = \frac{\partial u}{\partial z}\of{1,1,1} = -1, \frac{\partial f_2}{\partial z} = \frac{\partial v}{partial z}\of{1,1,1} = 0$.
\end{example}

\begin{proposition}[Implicit Differentiation]
    Let $F: \Omega_n \times \Omega_m \to \R^m$ be a $C^1$ function where $\Omega_n \subset \R^n$ and $\Omega_m \subset \R^m$ are open and let $c \in \R^m$. If $f: \Omega_n \to \Omega_m$ is a differentiable function such that $F\of{x, f\of{x}} = c$ for all $x \in \Omega_n$, then
    \[\frac{\partial F}{\partial y}\of{x, f\of{x}} D_f\of{x} = -\frac{\partial F}{\partial x}\of{x, f\of{x}}\]
    and
    \[D_f\of{x} = -\left[\frac{\partial F}{\partial y}\of{x, f\of{x}}\right]\inv \frac{\partial F}{\partial x}\of{x, f\of{x}}\]
    provided $\det \of{\frac{\partial F}{\partial y}\of{x, f\of{x}}} \neq 0$.
\end{proposition}
\begin{proof}
    Define $g: \Omega_n \to \Omega_n \times \Omega_m$ by $g\of{x, f\of{x}}$. Then $g$ is differentiable and 
    \[D_g \of{x} = \begin{bmatrix} I_n \\ D_f\of{x} \end{bmatrix}.\]
    Since $(F \circ g)\of{x} = c$, the chain rule yields $0 = D_{F \circ g}\of{x} = D_F\of{g\of{x}} D_g\of{x} = \begin{bmatrix} \frac{\partial F}{\partial x}\of{g\of{x}} & \frac{\partial F}{\partial y}\of{g\of{x}} \end{bmatrix} \begin{bmatrix} I_n \\ D_f\of{x} \end{bmatrix} = \frac{\partial F}{\partial x}\of{x, f\of{x}} + \frac{\partial F}{\partial y}\of{x, f\of{x}} D_f\of{x}$.

    Hence, the result.
\end{proof}