% camiregu 2024-jan-09
\chapter{the Laplace Transform}

%----------------------------------------------------------------------------------------

\begin{example}[transformations]
    Let's ponder for a moment some of the many fantastic transformations we've studied in the past:
    \begin{enumerate}
        \item $f(x) = x^2$
        \item $\frac{d}{dx}x^2 = 2x$
        \item $\int x^2 dx = \frac{1}{3}x^3 + C$
        \item $\int_{0}^{3} x^2 dx = g$
    \end{enumerate}
\end{example}

\begin{definition}[the Laplace Transform]
    For a function $f(t)$, $$\Lapl\set{f(t)} = \int_0^\infty f(t) e^{-st} dt, s>0$$
    where $\Lapl\set{f(t)}$ is a function of $s$.
\end{definition}

\begin{remark}
    $\Lapl$ provides an efficient method to solve ODEs.
\end{remark}

\begin{theorem}[Linearity of the Laplace transform]
    For $a,b$ constants, and $f, g$ functions, The Laplace transform satisfies
    $$\Lapl\set{af(t) + bg(t)} = a\Lapl\set{f(t)} + b\Lapl\set{g(t)}$$
\end{theorem}

\begin{definition}[the kernel]
    $e^{-st}$ is called the \textbf{kernel} of the Laplace transform.
\end{definition}

\begin{definition}[the domain]
    The \textbf{domain} of $F(s) = \Lapl\set{f(t)}$ is the set of values of $s$ such that $F(s)$ converges.
\end{definition}

\begin{example}[computing $\Lapl\set{t}$]
    \begin{recall}[integration by parts]
        $\int udv = uv - \int vdu$
    \end{recall}

    \begin{align*}
        \Lapl\set{t} &= \int_0^\infty t e^{-st} dt \\
        &= \lim_{\beta \to \infty} \left( \int_{0}^{\beta} te^{-st} dt \right) \\
        &= \lim_{\beta \to \infty} \left( \eval{t\left(-\frac{1}{s}\right) e^{st}}_0^\beta - \int_0^\beta \left(-\frac{1}{s} e^{-st} dt\right) \right) \\
        &= \lim_{\beta \to \infty} \left( -\frac{\beta}{s}e^{s\beta} + \frac{1}{s} \int_0^\beta e^{-st} dt \right) \\
        &= \lim_{\beta \to \infty} \left( -\frac{\beta}{s} e^{-s\beta} + \frac{1}{s} \left[-\frac{1}{s} e^{-st}\right]_0^\beta \right) \\
        &= \lim_{\beta \to \infty} \left( -\frac{\beta}{s} e^{-s\beta} + \frac{1}{s^2}\left[1 - e^{s\beta}\right] \right) \\
        &= \frac{1}{s^2}
    \end{align*}
    Wow, that computation sucked! Luckily, we will be given formula sheets to speed up these calculations. This is because the objective of this class is to solve ODEs using Laplace as a tool, not to practice computing integration by parts.
\end{example}

\begin{theorem}[Laplace transformations of power functions]
    We have, in general, $\Lapl\set{t^p} = \int_0^\infty t^p e^{-st} dt = \frac{1}{s^{p+1}} \int_0^\infty x^p e^-x dx$ where $x = st$. 
    So $$\Lapl\set{t^p} = \frac{1}{s^{p+1}} \Gamma (p+1)$$ 
\end{theorem}

\begin{theorem}[Properties of the Gamma function]
    \spacebeforelist
    \begin{enumerate}
        \item When $p > 0$, $\Gamma (p+1) = p\Gamma(p)$ (Recurrence Relation)
        \item When $p \in \N$, $\Gamma(p) = (p-1)!$ (Generalization of Factorial)
        \item When $p \leq 0$ and $p \in \Z$, $\Gamma(p)$ does not exist
        \item When $p < 0$ and $p \notin \Z$, use successive applications of the Recurrence Relation $\Gamma(p) = \frac{1}{p} \Gamma (p + 1)$
    \end{enumerate}
\end{theorem}