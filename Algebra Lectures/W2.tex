% camiregu 2024-jan-16
\chapter{Cyclic Subgroups}

%----------------------------------------------------------------------------------------

\begin{definition}[a cyclic group]
    A group $G$ is called \textbf{cyclic} if there is an element $a \in G$ such that $G = \set{a^j: j \in \Z}$. $a$ is called a \textbf{generator} of $G$. We indicate that $G$ is a cyclic group generated by $a$ with the notation $G = <a>$.
\end{definition}

\begin{theorem}
    Suppose $a \in G$. Then $<a>$ is a subgroup of $G$.
\end{theorem}
\begin{proof}
    Suppose $a^m, a^n \in <a>$ where $m, n \in \Z$. Then $a^ma^n = a^{m + n} \in <a>$ since $m + n \in \Z$. Also $a^{-m} \in <a>$ for all $m$ since $-m \in \Z$, and $a^ma^{-m} = a^0 = e = a^0 = a^{-m}a^m$. \\
    By the 2-step subgroup test $<a>$ is a subgroup.
\end{proof}

\begin{definition}[a cyclic subgroup]
    The subgroup $<a> \subseteq G$ is called the \textbf{cyclic subgroup} generated by $a \in G$.
\end{definition}

\begin{example}[generators]
    Take $G = \Z_6 = \set{0,1,2,3,4,5}$ together with addition mod 6. \\
    $\Z_6 = <1>$ since $n(1) = n \mod 6$. Note that we also have $\Z_6 = <5>$.
\end{example}

\begin{remark}
    In general, $\Z_n$ is cyclic and generated by $<-1>$. All finite cyclic are isomorphic to $Z_n$ for some $n$.
\end{remark}

\begin{remark}
    For $a \in G$, $<a> = <a\inv>$.
\end{remark}

\begin{example}[the integers]
    Take $G = \Z$. \\ 
    $<1> = \set{j1: j \in \Z} = \Z$. \\
    $<2> = \set{j2: j \in \Z} = \text{even numbers} \subset \Z$. \\
    $<m> = \set{jm: j \in \Z} = \text{integers divisible by } m$ for $m \neq 0$. \\
    $<0> = \set{0}$.
\end{example}

\begin{remark}
    Infinite cyclic groups are all isomorphic to $\Z$.
\end{remark}

\begin{definition}[the centre of a group]
    The \textbf{centre} of $G$ is the subset $$Z(G) = \set{x \in G: xa = ax \forall a \in G}$$ i.e., the elements that commute with everything in $G$.
\end{definition}

\begin{theorem}
    $Z(G)$ is a subgroup of $G$.
\end{theorem}
\begin{proof}
    Suppose $x,y \in Z(G) and a \in G$. Then $(xy)a = x(ya) = xay = axy = a(xy)$. Therefore $xy \in Z(G)$. \\
    Moreover, $xa=ax \implies x\inv xa = x\inv ax \implies a = x\inv ax \implies ax\inv = x\inv a xx\inv \implies ax\inv = x\inv a \implies x\inv \in Z(G)$. \\
    By the 2-step subgroup test, $Z(G)$ is a subgroup of $G$.
\end{proof}

\begin{remark}
    \begin{enumerate}
        \item $G$ is abelian $\iff$ $Z(G) = G$
        \item $Z(G)$ is abelian (even when $G$ is not)
        \item $Z(D_3) = \set{e}$ (brute force)
        \item $x \in Z(G) \iff x a x\inv = a$ for all $a \in G \iff a x a\inv = x$ for all $a \in G$
    \end{enumerate}
\end{remark}

\begin{example}[a non-trivial center]
    $Z(GL(2,\R)) = \set{\begin{bmatrix} a & 0 \\ 0 & a\end{bmatrix}: a \in \R^\times}$
\end{example}

\begin{definition}[the centralizer]
    Fix $b \in G$. The \textbf{centralizer} of $b$ in $G$ is \begin{align*}
        C_G(b) = C(b) &= \set{a \in G: ab = ba} \\
        &= \set{a \in G: aba\inv = b}
    \end{align*}
\end{definition}

\begin{theorem}
    For any $b \in G$, $C_G(b)$ is a subgroup.
\end{theorem}
\begin{proof}
    Subgroup test.
\end{proof}

\begin{example}
    \spacebeforelist
    \begin{enumerate}
        \item $C_G(e) = G$
        \item $C_G(b) = G \iff b \in Z(G)$
    \end{enumerate}
\end{example}