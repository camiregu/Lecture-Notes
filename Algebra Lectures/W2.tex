% camiregu 2024-jan-16
\chapter{Cyclic Subgroups}

%----------------------------------------------------------------------------------------

\begin{definition}[a cyclic group]
    A group $G$ is called \textbf{cyclic} if there is an element $a \in G$ such that $G = \set{a^j: j \in \Z}$. $a$ is called a \textbf{generator} of $G$. We indicate that $G$ is a cyclic group generated by $a$ with the notation $G = <a>$.
\end{definition}

\begin{theorem}
    Suppose $a \in G$. Then $<a>$ is a subgroup of $G$.
\end{theorem}
\begin{proof}
    Suppose $a^m, a^n \in <a>$ where $m, n \in \Z$. Then $a^ma^n = a^{m + n} \in <a>$ since $m + n \in \Z$. Also $a^{-m} \in <a>$ for all $m$ since $-m \in \Z$, and $a^ma^{-m} = a^0 = e = a^0 = a^{-m}a^m$. \\
    By the 2-step subgroup test $<a>$ is a subgroup.
\end{proof}

\begin{definition}[a cyclic subgroup]
    The subgroup $<a> \subseteq G$ is called the \textbf{cyclic subgroup} generated by $a \in G$.
\end{definition}

\begin{example}[generators]
    Take $G = \Z_6 = \set{0,1,2,3,4,5}$ together with addition mod 6. \\
    $\Z_6 = <1>$ since $n(1) = n \mod 6$. Note that we also have $\Z_6 = <5>$.
\end{example}

\begin{remark}
    In general, $\Z_n$ is cyclic and generated by $<-1>$. All finite cyclic are isomorphic to $Z_n$ for some $n$.
\end{remark}

\begin{remark}
    For $a \in G$, $<a> = <a\inv>$.
\end{remark}

\begin{example}[the integers]
    Take $G = \Z$. \\ 
    $<1> = \set{j1: j \in \Z} = \Z$. \\
    $<2> = \set{j2: j \in \Z} = \text{even numbers} \subset \Z$. \\
    $<m> = \set{jm: j \in \Z} = \text{integers divisible by } m$ for $m \neq 0$. \\
    $<0> = \set{0}$.
\end{example}

\begin{remark}
    Infinite cyclic groups are all isomorphic to $\Z$.
\end{remark}

\begin{definition}[the centre of a group]
    The \textbf{centre} of $G$ is the subset $$Z(G) = \set{x \in G: xa = ax \forall a \in G}$$ i.e., the elements that commute with everything in $G$.
\end{definition}

\begin{theorem}
    $Z(G)$ is a subgroup of $G$.
\end{theorem}
\begin{proof}
    Suppose $x,y \in Z(G) and a \in G$. Then $(xy)a = x(ya) = xay = axy = a(xy)$. Therefore $xy \in Z(G)$. \\
    Moreover, $xa=ax \implies x\inv xa = x\inv ax \implies a = x\inv ax \implies ax\inv = x\inv a xx\inv \implies ax\inv = x\inv a \implies x\inv \in Z(G)$. \\
    By the 2-step subgroup test, $Z(G)$ is a subgroup of $G$.
\end{proof}

\begin{remark}
    \begin{enumerate}
        \item $G$ is abelian $\iff$ $Z(G) = G$
        \item $Z(G)$ is abelian (even when $G$ is not)
        \item $Z(D_3) = \set{e}$ (brute force)
        \item $x \in Z(G) \iff x a x\inv = a$ for all $a \in G \iff a x a\inv = x$ for all $a \in G$
    \end{enumerate}
\end{remark}

\begin{example}[a non-trivial center]
    $Z(GL(2,\R)) = \set{\begin{bmatrix} a & 0 \\ 0 & a\end{bmatrix}: a \in \R^\times}$
\end{example}

\begin{definition}[the centralizer]
    Fix $b \in G$. The \textbf{centralizer} of $b$ in $G$ is \begin{align*}
        C_G(b) = C(b) &= \set{a \in G: ab = ba} \\
        &= \set{a \in G: aba\inv = b}
    \end{align*}
\end{definition}

\begin{theorem}
    For any $b \in G$, $C_G(b)$ is a subgroup.
\end{theorem}
\begin{proof}
    Subgroup test.
\end{proof}

\begin{remark}
    \begin{enumerate}
        \item $C_G(e) = G$
        \item $C_G(b) = G \iff b \in Z(G)$
        \item $e \in C_G\of{b}$, $<b> \subseteq C_G\of{b}$
    \end{enumerate}
\end{remark}

\begin{example}[a centralizer]
    $C_{GL(2, \R)}\of{\begin{bmatrix} 1 & 0 \\ 0 & -1 \end{bmatrix}} = \set{\begin{bmatrix} a & 0 \\ 0 & b \end{bmatrix}: a, b \in \R^\times}$
\end{example}

\begin{recall}
    $G$ is cyclic if $G = <a> = \set{a^j: j \in \Z}$ for some $a \in G$.
\end{recall}

\begin{theorem}
    Suppose $a \in G$. Then 
    \begin{enumerate}
        \item If $\ord{ a } = \infty$, then $a^k = a^j \iff j = k$
        \item If $\ord{ a } = n$, then $a^k = a^j \iff$ $n$ divides $k - j$
    \end{enumerate}
\end{theorem}
\begin{proof}
    \begin{enumerate}
        \item Suppose $\ord{ a } = \infty$. This means $a^n \neq e$ for any $n \geq 1$. Suppose now $a^k = a^j$ with $k \geq j$. Then $a^ka^{-j} = aja^{-j} = e \implies a^{k-j}$ for $k - j \geq 0$. Since $a^n \neq e \forall n \geq 1$, we have $k - j = 0 \implies k = j$.
        \item Suppose $\ord{ a } = n$. This means $a^n = e$ and $n$ is the least positive number satisfying this equation. Suppose $a^k = a^j$ with $k \geq j$. Then $a^{k-j} = e$ where $k - j \geq 0$. By definition of $n$, $n \leq k - j$. By the division algorithm, $k - j = qn + r$ where $q, r \in \Z$ are unique and $0 \leq r \leq n - 1$. \\
        $e = a^{k - j} = a^{qn + r} = a^{qn} a^r = \left(a^n\right)^q a^r = e^q a^r = ea^r =a^r$, so $r = 0$ by the minimality of $n$, and so $k - j = qn \implies \frac{k-j}{n} = q \in Z \implies n$ divides $k - j$.\\
        Conversely if $qn = k - j$, then $a^{k-j} = \left(a^n\right)^q = e^q = e \implies a^k = a^j$.
    \end{enumerate}
\end{proof}

\begin{remark}
    In part 2., $n$ divides $k-j \iff (k-j) \mod n = 0 \iff k \mod n = j \mod n$
\end{remark}

\begin{corollary}
    Suppose $\ord{ a } n$. Then $a^k = e$ for some $k \in \Z \iff$ k is a multiple of $\ord{ a }$
\end{corollary}
\begin{proof}
    Suppose $a^k = e$. Then $a^k = a^0$, so $n$ divides $k - 0 = k$.
\end{proof}

\begin{corollary}
    Suppose $a \in G$. Then \begin{enumerate}
        \item If $\ord{a} = n$ then $<a> = \set{e, a^1, a^2, \dots, a^{n-1}}$ and $\ord{<a>} = \ord{a}$.
        \item If $\ord{a} = \infty$, then $<a>$ is infinite and $\ord{<a>} = \ord{a} = \infty$
    \end{enumerate}
\end{corollary}
\begin{proof}
    Didn't take notes for this one.
\end{proof}

\begin{corollary}
    Suppose $G$ is a finite group and $a,b \in G$. Then \begin{enumerate}
        \item $\ord{a}, \ord{b}$ are finite
        \item If $ab = ba$ then $\ord{ab}$ divides $\ord{a} \ord{b}$
    \end{enumerate}
\end{corollary}
\begin{proof}
    \begin{enumerate}
        \item Suppose by way of contradiction that $\ord{a}$ is infinite. Then $<a> \subseteq G$ is infinite. But $G$ is finite so $\ord{<a>} \leq \ord{G}$ is a contradiction.
        \item $(ab)^{\ord{a}\ord{b}} = a^{\ord{a}\ord{b}}b^{\ord{a}\ord{b}} = (a^{\ord{a}})^{\ord{b}}(b^{\ord{b}})^{\ord{a}} = e^{\ord{b}}e^{\ord{a}} = e$
    \end{enumerate}
\end{proof}

2 examples omitted. Sorry, I'm prepping for my tutorial later! 

\begin{theorem}
    Suppose $a \in G$ and $\ord{a} = n$. Then for any $k \geq 1$, $<a^k> = <a^{\gcd \of{n,k}}>$ and $\ord{a^k} = \frac{n}{\gcd \of{n, k}}$
\end{theorem}

\begin{theorem}[Fundamental Theorem of Cyclic Groups]
    Suppose $G = <a>$ is cyclic and $\ord{G} = n$. Then \begin{enumerate}
        \item Every subgroup of $H$ is cyclic and $k = \ord{H}$ divides $n = \ord{G}$, i.e., $k$ is a divisor of $n$
        \item For every divisor $k$ of $n$, there is a unique subgroup of $G$ of order $k$ and it is equal to $<a^{\frac{n}{k}}>$
    \end{enumerate}
\end{theorem}
\begin{proof}
    \begin{enumerate}
        \item Suppose $H$ is a subgroup of $G$ and $H \neq <e>$. Let $m \geq 1$ be the least power of $a$ such that $a^m \in H$. Since $H$ is closed under multiplication and inversion, $<a^m> \subseteq H$. Suppose $a^j \in H$. By the division algorithm, $j = qm + r$ with $0 \leq r \leq m$ $\implies a^j = \left(a^m\right)^q a^r \implies a^j \left(a^m\right)^{-q} = a^r$, so since $a^j,\left(a^m\right)^{-q} \in H$, $a^r \in H \implies r = 0$ by the minimality of $m$.
        \item Suppose $k$ divides $n$, i.e. $\frac{n}{k}$ is an integer. Recall that $\ord{<a^\frac{n}{k}>} = \ord{a^\frac{n}{k}} = k$. It follows that $\ord{<a^\frac{n}{k}>} = k$.
        
        Suppose $H \subseteq <a>$ is a subgroup and $\ord{H} = k$. By part 1, $H = a^m$ for some $m \geq 1$. By Theorem 2.8, $k = \ord{H} = \ord{<a^m>} = \ord{a^m} = \frac{n}{\gcd\of{m,n}} \implies \gcd\of{m,n} = \frac{n}{k}$.

        By Theorem 2.8 again, $H = <a^m> = <a^{\gcd\of{m,n}}> = <a^\frac{n}{k}>$.
    \end{enumerate}
\end{proof}

\newpage

\begin{example}[the subgroups of $\Z_{12}$]
    The divisors of $n = 12$ are $1,2,3,4,6,12$
    \begin{itemize}
        \item $k = 1$: $<0>$
        \item $k = 2$: $<6> = \set{0,6}$
        \item $k = 3$: $<4> = \set{0,4,8}$
        \item $k = 4$: $<3> = \set{0,3,6,9}$
        \item $k = 6$: $<2> = \set{0,2,4,6,8,10}$
        \item $k = 12$: $\Z_{12} = \set{0,1,2,3,4,5,6,7,8,9,10,11}$
    \end{itemize}
    \begin{note}
        The lattice of subgroups of $\Z_{12}$ illustrates the containment relationships.
        \insertimage[0.5]
    \end{note}
\end{example}

\begin{remark}
    In $\Z_n$, clearly $<m> \subseteq <k> \iff m \in <k> \iff ka = m \iff k \text{ divides } m$.
\end{remark}

\begin{example}[subgroups of $\Z_p$]
    Consider $\Z_p$ where $p$ is prime. The only subgroup of $\Z_p$ is $<0>$.
\end{example}