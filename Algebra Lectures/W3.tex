% camiregu 2024-jan-23
\chapter{Permutation Groups (Symmetric Groups)}

%----------------------------------------------------------------------------------------

\begin{definition}[the Euler $\phi$-function]
    The Euler $\phi$-function is defined for every positive integer $d \geq 1$ by
    \[\phi\of{d} = \left\{ \begin{matrix} 1 \text{ if } d = 1 \\ \end{matrix} \right.\]
\end{definition}

\begin{definition}
    Suppose $A \neq \phi$ is a set. A \textbf{permutation} of $A$ is a bijection $\beta: A \to A$ (1-1, onto). The \textbf{permutation group} (\textbf{symmetric group}) of $A$ is the set of permutations of $A$ under composition.
\end{definition}

\begin{recall}[some facts about functions]
    Let $S_A$ be the symmetric group of $A \neq \phi$.

    If $\alpha, \beta \in S_A$ then $\alpha \circ \beta \of{a} = \alpha\of{\beta\of{a}}$ for all $a \in A$.

    From MATH1800 composition of 1-1 and onto functions is again 1-1 and onto, i.e., $\alpha \circ \beta \in S_A$.

    From MATH1800 $(\alpha \circ \beta) \circ \gamma = \alpha \circ (\beta \circ \gamma)$ for all $\alpha, \beta, \gamma \in S_A$.
    $\alpha$ permutation $\iff$ $\alpha$ is invertible under composition.s
\end{recall}

\begin{remark}
    Define $e \in S_A$ by $e\of{a} = a$ for all $a \in A$. Clearly $e \circ \alpha \of{a} = e\of{\alpha\of{a}} = \alpha\of{a}$ for all $a \in A \implies e \circ a = a$.
    We see that $S_A$ truly is a group.
\end{remark}

\begin{example}
    Take $A = \set{1,2,3}$.
    What are the permutations in $S_3 = S_A$?
    \begin{itemize}
        \item $e \in S_3: e\of{1} = 1, e\of{2} = 2, e\of{3} = 3$.
        \item $\beta \in S_3$ where $\beta\of{1} = 2, \beta\of{2} = 3, \beta\of{3} = 1$.
    \end{itemize}
    Let's rewrite $\beta$ as follows: $\begin{bmatrix} 1 & 2 & 3 \\ 2 & 3 & 1 \end{bmatrix} = \R$.

    In general for any $\alpha \in S_3$, we may rewrite it as $\begin{bmatrix} 1 & 2 & 3 \\ \alpha\of{1} & \alpha\of{2} & \alpha\of{3} \end{bmatrix} = \R$.

    The number of permutations is given by the number of choices. This is $3! = 3 \cdot 2 \cdot 1$. We just proved that $\ord{S_3} = 3! = 6$.
    
    Similar reasoning tells us that $\ord{S_n} = n!$ for every $n \geq 1$.
\end{example}

\begin{question}
    Paul Mezo said we "know everything" about linear algebra. What does that mean?
\end{question}
\begin{answer}
    There are no unsolved problems in finite linear algebra.
\end{answer}

\section{Cycle Notation}

Consider $S_3$ and $\begin{bmatrix} 1 & 2 & 3 \\ 2 & 3 & 1 \end{bmatrix} \in S_3$. We rewrite this permutation as follows: $(1 2 3)$.

Notice that $\alpha = (123) \neq (132) = \beta$, but they are both 3-cycles.

Also, $\gamma = (12)$ is the permutation such that $\gamma\of{1} = 2, \gamma\of{2} = 1, \gamma\of{3} = 3$. It's a 2-cycle.

We omit 1-cycles.

The six permutations in $S_3$ in cycle notation are $e, (12), (13), (23), (123), (132)$.

\begin{example}[cycles of $S_4$]
    Consider $S_4$. $\ord{S_4} = 24 = 4!$.
    \begin{itemize}
        \item $e$,
        \item $(12), (13), (14), (23), (24), (34)$
        \item $(123), (134), \dots$
        \item $(1234), (1243), \dots$
        \item $(12)(34), (13)(24), (14)(23)$
    \end{itemize}
\end{example}

\begin{definition}[disjoint cycles]
    Two cycles $(a_1, a_2, \dots, a_m), (b_1, b_2, \dots, b_k) \in S_n$ are \textbf{disjoint} if $a_j \neq b_l$ for any $j, l$. Their product can be written equally in either order.
\end{definition}

Composition of permutations is interpreted as products of cycles as follows:

\begin{example}[compositions in $S_7$]
    \begin{itemize}
        \item $(623)(12) = (1362)$
        \item $(12)(347)(23) = (24731)$
    \end{itemize}
\end{example}