% camiregu 2024-jan-09
\chapter{Introduction to Groups}

%----------------------------------------------------------------------------------------
\begin{definition}[a group]
    A \textbf{group} $G$ is a nonempty set together with a multiplication $G \times G \to G$ satisfying
    \begin{enumerate}
        \item $(ab)c = a(bc) \forall a, b, c, \in G$, (Associativity)
        \item there exists $e \in G$ such that $ea = ae = a \forall a \in G$, (Identity)
        \item and for every $a \in G$ there exists $b \in G$ such that $ab = ba = e$. (Inverse)
    \end{enumerate}
\end{definition}

\begin{example}[a group]
    Let $\R^* = \R^\dag = \set{a \in \R: a \neq 0}$ together with multiplication on $\R$. \\
    Associativity is immediate. \\
    The identity is $1 \in \R^*$. \\
    For every $a \in \R^*$, $\frac{1}{a} \in \R$ and $a(\frac{1}{a}) = \frac{1}{a}(a) = 1$. \\
    So $\R^*$ is a group.
\end{example}

\begin{remark}
    When we need to highlight the group multiplication we write a group as a pair of the set and the multiplication, e.g., $(\R, +), (\R, \cdot)$. \\
    From now on, $G$ is \textbf{always} a group.
\end{remark}

\begin{theorem}
    There is a unique identity element in G.
\end{theorem}

\begin{theorem}[Cancellation] \label{thm:Cancellation}
    Suppose $ba = ca$ for $a,b,c \in G$. Then $b = c$ 
\end{theorem}
\begin{proof}
    Let $d \in G$ be an inverse for $a$, i.e. $da = ad = e$.
    Multiplying on the right by $d$, we obtain
    \begin{align*}
        (ba)d = (ca)d &\implies b(ad) = c(ad) \\
        &\implies be = ce \\
        &\implies b = c.
    \end{align*}
\end{proof}   

\begin{theorem}[Uniqueness of Inverses]
    For every $a \in G$ there is a unique element $a\inv \in G$ such that $a a\inv = a\inv a = e$.
\end{theorem}
\begin{proof}
    Suppose $a \in G$ and $b, b' \in G$ are inverses of $a$, then 
    $$ba = e = b'a \implies b = b'$$ (by \cref{thm:Cancellation})
\end{proof}

\begin{example}[inverses in different groups]
    \spacebeforelist
    \begin{enumerate}
        \item For $b \in \R^*$, $b\inv = \frac{1}{b}$.
        \item For $b \in \R$ under addition $b\inv = -b$.
        \item For $b \in \Z_n$, $b\inv = n - b$.
    \end{enumerate}
\end{example}

\begin{example}[groups using a field $F$]
    \spacebeforelist
    \begin{enumerate}
        \item $(F, +)$ is a group (Imitate $(\R, +)$).
        \item $(F^*, \cdot)$ where $F^* = F^\dag = \set{a \in F: a \neq 0}$ is a group. In particular, if $p$ is a prime number, then $\Z_p^* = \set{1, \dots, p - 1}$ is a group.
        \item The set of $m \times n$ matrices with entries in $F$, $M_{mn}(F)$ is a group under addition. When $n=1$, $M_{m1}(F) = F^m$.
        \item The set of invertible $m \times n$ matrices with entries in $F$, $GL(n,F) = \set{A \in M_{mn}(F): \mathrm{det}(A) \neq 0}$ together with matrix multiplication is called (rank $n$) \textbf{general linear group} (over $F$). The identity matrix $I \in GL(n, F)$ is the identity. $\mathrm{det}(A) \neq 0 \implies \exists A\inv \in GL(n,F)$ such that $A A\inv = A\inv A = I$.
    \end{enumerate}
\end{example}

\begin{example}[the symmetries of the equilateral triangle]
    Let $\sigma = \text{flip through the vertical axis}$. Let $\rho = \text{rotation by } \frac{2\pi}{3}$. \\
    We can compose two symmetries, e.g., $\sigma\rho = \sigma \cdot \rho$. \\
    We can show that the symmetries given by $\sigma$ and $\rho$ under composition are $\set{e, \rho, \rho^2, \sigma, \sigma\rho, \sigma\rho^2}$ where $e = \text{doing nothing}$. \\
    We call this set $D_3$. It forms a group under composition. Clearly $\rho^3 = \rho\rho\rho = e$, $\sigma^2 = \sigma\sigma = e$, and $\sigma\rho\sigma = \rho^2 = \rho\inv$.
\end{example}

\begin{definition}[a dihedral group]
    The \textbf{dihedral group} of order $2n$ is defined by $$D_n = \set{e, \rho, \dots, \rho^{n-1}, \sigma, \sigma\rho, \dots, \sigma\rho^{n - 1}}$$ where $p^n = e$, $\sigma^2 = e$, and $\sigma\rho\sigma = \rho\inv$.
    This is a group with the multiplication given by $\sigma\rho\sigma = \rho\inv$.
\end{definition}

\begin{remark}
    $D_n$ is the group of symmetries of a regular n-gon.
\end{remark}

\begin{definition}[an Abelian Group]
    A group $G$ is \textbf{abelian (commutative)} if $ab=ba$ for all $a, b \in G$
\end{definition}

\begin{example}[classifying groups]
    \spacebeforelist
    \begin{enumerate}
        \item $(F, +)$ where $F$ is a field is Abelian.
        \item $(F^*, \cdot)$ where $F$ is a field is Abelian.
        \item $(M_{mn}(F), +)$ is Abelian.
        \item $(GL(n, F), \cdot)$ is not Abelian.
        \item $D_n$ is not Abelian.
    \end{enumerate}
\end{example}

\begin{definition}[the group of units]
    Let $n \geq 2$ and $U(n) = \set{1 \leq k \leq n-1: \gcd (k,n) = 1}$. \\
    $U(n)$ is called the \textbf{group of units} of $\Z_n$
\end{definition}

\begin{recall}[Facts about $d = \gcd (a,b)$]
    .
    \begin{enumerate}
        \item $d \divides a$ and $d \divides b$, and $d$ is the largest integer with this property
        \item There exists $l, m \in \Z$ such that $\gcd (a,b) = la + mb$
        \item $\gcd (a,b)$ is the smallest positive $\Z$-linear combination of $a$ and $b$.
        \item If $f \divides a$ and $f \divides b$ then $f$ divides $\gcd (a,b) = la + mb \implies f \divides d$
    \end{enumerate}
\end{recall}

\begin{example}[$U(n)$ together with multiplication $\mod n$ is a group]
    Facts 2 and 3 tell us that $\gcd (k,n) = 1 \iff \exists l,m \in \Z$ such that $lk + mn = 1$. \\
    So $U(2) = \set{1}, U(3) = \set{1,2}, U(4) = \set{1, 3}, U(5) = \set{1,2,3,4}$, etc. \\
    So $U(p) = \set{1, \dots, p-1} = \Z_p^*$ where $p$ is prime.
\end{example}

\begin{definition}[exponentiation]
    Suppose $g \in G$.
    \begin{enumerate}
        \item $g^0 = e$
        \item $g^n = g \cdot \cdots \cdot g$ ($n$ times)
        \item $g^{-n} = (g \inv)^n$
    \end{enumerate}
\end{definition}

\begin{theorem}[Socks and Shoes]
    Suppose $a,b \in G$. Then $(ab)\inv = b\inv a\inv$ (only relevant for non-abelian groups)
\end{theorem}
\begin{proof}
    \begin{align*}
        (ab)(b\inv a\inv) = aea\inv &= aa\inv = e \\
        (b\inv a\inv)(ab) = b\inv e b &= b\inv b = e
    \end{align*}
\end{proof}

\begin{definition}[the order of a group and its elements]
    The number of elements in $G$ is called the \textbf{order} of $G$. Suppose $a \in G$. Then the \textbf{order of a} is the largest positive integer $n$ such that $a^n = e$. If no such integer exists, we say $a$ has \textbf{infinite order}. We denote the order of $a$ by $\ord{a}$.
\end{definition}

\begin{example}[the order of $\set{e}$]
    We know $\ord{\set{e}} = 1$, and $e^1 = e \implies \ord{e} = 1$
\end{example}

\begin{example}[the order of $\R^*$]
    $\R^*$ is an infinite group so it has infinite order. \\
    Obviously, $\ord{1} = 1$. \\
    $\ord{-1} = 2$ since $(-1)^2 = 1$ and $(-1)^1 \neq 1$. \\ 
    All other real numbers in $\R^*$ have infinite order.
\end{example}

\begin{example}[the order of $D_3$]
    $\ord{D_3} = 6$. \\
    $\ord{\sigma}=2, \ord{\rho}=3, \ord{\rho^2}=3, \ord{\sigma\rho}=2, \ord{\sigma\rho^2}=2$.
\end{example}

\begin{definition}[a subgroup]
    A \textbf{subgroup} of $G$ is a subset $H \subseteq G$ which is a group under the same group multiplication as $G$.
\end{definition}

\newpage % for prettiness
\begin{example}[subgroups]
    \spacebeforelist
    \begin{enumerate}
        \item $\set{\pm 1} \subseteq \R^*$ is a subgroup
        \item $\Z_5 \subseteq \Z$ is not a subgroup of $\Z$ since they have different group multiplications
    \end{enumerate}
\end{example}

\begin{theorem}[2-step subgroup test]
    Suppose $H$ is a non-empty subset of $G$. Then $H$ is a subgroup of $G$ if and only if:
    \begin{enumerate}
        \item $a, b \in H \implies ab \in H$ (closure under multiplication)
        \item $a \in H \implies a\inv \in H$ (closure under inverse)
    \end{enumerate}
\end{theorem}

\begin{theorem}[1-test subgroup test]
    $\emptyset \neq H \subseteq G$ is a subgroup $\iff$ $a,b \in H \implies ab\inv \in H$
\end{theorem}

\begin{proof}
    The forward direction is immediate. \\
    "$\impliedby$" Suppose 1 and 2 hold. 1 tells us that the group multiplication on $G$ restricts to a multiplicationon $H$. The associativity of this multiplication on $H$ is inherited from the associativity of the group multiplication on $G$. \\
    By 1 and 2, for any $a \in H$, $a\inv in H$ and $e = a a\inv \in H$. Therefore $e \in H$. \\
    Finally, 2 is the inverse axiom for $H$.
\end{proof}

\begin{example}[showing subgroup-ness]
    Let $\mu_4 = \set{a \in \C^*: a^4 = 1} = \set{1, -1, i, -i}$. \\
    $\mu_4 \neq \emptyset$. \\
    $a,b \in \mu_4 \implies (ab)^4 = a^4 b^4 = (1)(1) = 1 \implies ab \in \mu_4$ \\
    $a \in \mu_4 \implies (a\inv)^4 = a^{-4} = (a^4)\inv = 1\inv = 1 \implies a\inv \in \mu_4$
\end{example}