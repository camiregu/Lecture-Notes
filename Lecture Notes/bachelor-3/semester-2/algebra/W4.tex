% camiregu 2024-jan-30
\chapter{Isomorphisms}

%----------------------------------------------------------------------------------------

\begin{remark}
    \begin{itemize}
        \item Cayley's Theorem says that every finite group is isomorphic to a subgroup of some $S_n$.
        \item Historically, the idea of a group comes from work with $S_n$.
    \end{itemize}
\end{remark}

\begin{example}
    Recall $D_3 = \set{e, \rho, \rho^2, \sigma, \sigma\rho, \sigma\rho^2}$

    Note that $S_3$ also has order 6.

    We may identify the elements in $D_3$ by how they permute the vertices of a triangle:
    \begin{itemize}
        \item $e \leftrightarrow e$
        \item $\rho \leftrightarrow \cyc{1,2,3}$
        \item $\rho^2 \leftrightarrow \cyc{1,3,2}$
        \item $\sigma \leftrightarrow \cyc{1,3}$
        \item $\sigma\rho \leftrightarrow \cyc{1,2}$
        \item $\sigma\rho \leftrightarrow \cyc{2,3}$
    \end{itemize}

    If we define $\phi: D_3 \to S_3$ by $\phi\of{e} = e, \phi\of{\rho} = \cyc{1,2,3}$ etc. as above, then it remains to show that $\phi\of{ab} = \phi\of{a} \phi\of{b}$ for all $a,b \in D_3$.

    After a brute force check, we can verify that the above holds.
\end{example}

\begin{definition}[an isomorphism]
    Suppose $G$ and $\bar{G}$ are groups. An \textbf{isomorphism} is a map $\phi:G \to \bar{G}$ which is bijective and $\phi\of{ab} = \phi\of{a} \phi\of{b}$ for all $a,b \in G$.
\end{definition}

\begin{example}
    Let $G = \cygr{a} = \set{a^j: j \in \Z}$ be an infinite cyclic group. ($\ord{a} = \infty$).

    Define $\phi: \Z \to G$ by $\phi(j) = a^j$.

    This function is bijective. Proof omitted because I'm sleepy! It's not too hard, do it as an exercise.

    Finally, $\phi\of{j+k} = a^{j+k} = a^j a^k = \phi\of{j} \phi\of{k}$. This proves that $\phi: \Z \to \cygr{a}$ is an isomorphism.
\end{example}

\begin{definition}[isomorphic groups]
    We say groups $G$ and $\bar{G}$ are \textbf{isomorphic} if there is an isomorphism $\phi: G \to \bar{G}$. In this case we write $G \cong \bar{G}$.
\end{definition}

\begin{theorem}
    Suppose $\phi: G \to \bar{G}$ is a group isomorphism. Then 
    \begin{enumerate}
        \item $\phi\of{e} = \bar{e}$ is the identity in $\bar{G}$
        \item $\phi\of{b^n} = \left(\phi\of{b}\right)^n$ for all $b \in G$
        \item $ab = ba$ in $G \implies \phi\of{a} \phi\of{b} = \phi\of{b} \phi\of{a}$ in $\bar{G}$
        \item $G = \cygr{b} \implies \bar{G} = \cygr{\phi\of{b}}$
        \item $\ord{b} = \ord{\phi\of{b}}$ for all $b \in G$
        \item Omitted.
        \item $\ord{G} = \ord{\bar{G}}$ (In particular $G$ finite $\iff$ $\bar{G}$ finite)
    \end{enumerate}
\end{theorem}
\begin{proof}
    Sketch of proof\begin{enumerate}
        \item $\phi\of{e} = \phi\of{ee} = \phi\of{e} \phi\of{e} \implies \bar{e} = \bar{e}\phi\of{e} = \phi\of{e}$.
        \item Prove by induction on $n \geq 1$. For $n \leq -1$, replace $b$ by $b\inv \in G$.
        \item $ab = ba \in G \implies \phi\of{a} \phi\of{b} = \phi\of{ab} = \phi\of{ba} = \phi\of{b} \phi\of{a}$
        \item Suppose $G = \cygr{b}$. Let $a \in \Z$. Since $\phi$ is a bijection, there exists a unique $a \in G$ such that $\phi\of{a} = \bar{a}$. Since $G = \cygr{b}$, $a = b^j$ for some $j \in \Z$. By 2, $\bar{a} = \phi\of{a} = \phi\of{b^j} = \phi\of{b}^j \implies \bar{a} \in \cygr{\phi\of{b}} \implies \bar{G} \subseteq \cygr{\phi\of{b}} \implies \bar{G} = \cygr{\phi\of{b}}$.
        \item Suppose $\ord{b} = n < \infty$. Then $\phi\of{b}^n = \phi\of{b^n} = \phi\of{e} = \bar{e}$. $n$ must be the lowest of these, otherwise we arrive at $b^m = e$ for $m < n$ is a contradiction. A similar proof by contradiction is reached if $\ord{b} = \infty$.
    \end{enumerate}
\end{proof}

\begin{example}
    Define $\mu_n = \set{e^{2\pi i \frac{k}{n}}: 0 \leq k \leq n - 1} = \set{z \in \C^\times: z^n = 1} \subseteq \C^\times$.

    $\mu_n$ is a subgroup of $\C^\times$.

    Define $\phi: \Z_n \to \mu_n$ by $\phi\of{k} = e^{2\pi i \frac{k}{n}}$ for all $0 \leq k \leq n - 1$.
    
    Exercise: $\phi$ is a bijection.
\end{example}

\begin{note}
Examples omitted from lecture today; sorry I'm sleepy and need to do tutorial prep.
\end{note}

\begin{theorem}
    Suppose $\phi: G \to \bar{G}$ is a group isomorphism. Then \begin{enumerate}
        \item $\phi\inv: \bar{G} \to G$ is an isomorphism
        \item $G$ abelian $\iff$ $\bar{G}$ is abelian
        \item $G$ cyclic $\iff \bar{G}$ is cyclic
        \item $K$ subgroup of $G \implies \phi\of{K} = \set{\phi\of{k}: k \in K}$ subgroup of $\bar{G}$
        \item $\bar{K}$ subgroup of $\bar{G} \implies \phi\inv\of{\bar{K}}$ subgroup of $G$. 
        \item $\phi\of{Z\of{G}} = Z\of{\bar{G}}$
    \end{enumerate}
\end{theorem}
\begin{proof}
    \begin{enumerate}
        \item $\phi\inv: \bar{G} \to G$ exists and is a bijection since $\phi$ is a bijection. Must prove $\phi\inv\of{\bar{a}\bar{b}} = \phi\inv\of{\bar{a}} \phi\inv\of{\bar{b}}$ for $\bar{a},\bar{b}, \in \bar{G}$. Since $\phi$ is a bijection, there exist unique $a,b \in G$ such that $\phi\of{a} = \bar{a}$ and $\phi\of{b} = \bar{b}$. So \[\phi\inv\of{\bar{a}\bar{b}} = \phi\inv\of{\phi\of{a}\phi\of{b}} = \phi\inv\of{\phi\of{ab}} = ab = \phi\inv\of{\bar{a}} = \phi\inv\of{\bar{b}}\]
        \item Let's prove $G$ abelian $\iff \bar{G}$ abelian. Suppose $G$ is abelian, and $\bar{a},\bar{b} \in \bar{G}$. Let $a,b, \in G$ such that $\phi\of{a} = \bar{a}, \phi\of{b} = \bar{b}$, then \[\bar{a}\bar{b} = \phi\of{a} \phi\of{b} = \phi\of{ab} = \phi\of{ba} = \phi\of{b} \phi\of{a} = \bar{b}\bar{a}.\] This proves $\bar{G}$ is abelian/
        \item Was done previously
        \item Suppose $K \subseteq G$ is a subgroup. Suppose $\phi\of{K}, \phi\of{k'} \in \phi\of{K}$ where $k, k' \in K$. Then $\left(\phi\of{k}\right)\inv \phi\of{k'} = \phi\of{k\inv} \phi\of{k'} = \phi\of{k\inv k'}$. Since $k\inv k' \in K$, $\phi\of{k\inv k'} = \left(\phi\of{k}\right)\inv \phi\of{k'} \in \phi\of{K}$
        \item Follows from 4
        \item First we prove $\phi\of{Z\of{G}} \subseteq Z\of{\bar{G}}$. Suppose $z \in Z\of{G}$ and $\bar{a} \in \bar{G}$. Let $a \in G$ such that $\phi\of{a} = \bar{a}$. Then $\phi\of{z} \bar{a} = \phi\of{z}\phi{z} = \phi\of{za} = \phi\of{az} = \phi\of{a} \phi\of{z} = \bar{a} \phi\of{z} \implies \phi\of{z} \in Z\of{\bar{G}} \implies \phi\of{Z\of{G}} \subseteq Z\of{\bar{G}}$. By symmetry, $Z\of{\bar{G}} \subseteq \phi\of{Z\of{G}}$ so they are equal.
    \end{enumerate}
\end{proof}

\begin{definition}[an automorphism]
    An \textbf{automorphism} of $G$ is an isomorphism $\phi: G \to G$. The set of automorphisms of $G$ is denoted by $\Aut\of{G}$.
\end{definition}

\begin{theorem}
    $\Aut\of{G}$ is a group under composition of functions.
\end{theorem}