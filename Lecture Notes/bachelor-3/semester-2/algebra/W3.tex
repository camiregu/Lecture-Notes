% camiregu 2024-jan-23
\chapter{Permutation Groups (Symmetric Groups)}

%----------------------------------------------------------------------------------------

\begin{definition}[the Euler $\phi$-function]
    The Euler $\phi$-function is defined for every positive integer $d \geq 1$ by
    \[\phi\of{d} = \left\{ \begin{matrix} 1 \text{ if } d = 1 \\ \ord{\set{1 \leq j \leq d - 1: \gcd\of{j,d} = 1}} \end{matrix} \right.\]
\end{definition}

\begin{definition}
    Suppose $A \neq \phi$ is a set. A \textbf{permutation} of $A$ is a bijection $\beta: A \to A$ (1-1, onto). The \textbf{permutation group} (\textbf{symmetric group}) of $A$ is the set of permutations of $A$ under composition.
\end{definition}

\begin{recall}[some facts about functions]
    Let $S_A$ be the symmetric group of $A \neq \phi$.

    If $\alpha, \beta \in S_A$ then $\alpha \circ \beta \of{a} = \alpha\of{\beta\of{a}}$ for all $a \in A$.

    From MATH1800 composition of 1-1 and onto functions is again 1-1 and onto, i.e., $\alpha \circ \beta \in S_A$.

    From MATH1800 $(\alpha \circ \beta) \circ \gamma = \alpha \circ (\beta \circ \gamma)$ for all $\alpha, \beta, \gamma \in S_A$.
    $\alpha$ permutation $\iff$ $\alpha$ is invertible under composition.s
\end{recall}

\begin{remark}
    Define $e \in S_A$ by $e\of{a} = a$ for all $a \in A$. Clearly $e \circ \alpha \of{a} = e\of{\alpha\of{a}} = \alpha\of{a}$ for all $a \in A \implies e \circ a = a$.
    We see that $S_A$ truly is a group.
\end{remark}

\begin{example}
    Take $A = \set{1,2,3}$.
    What are the permutations in $S_3 = S_A$?
    \begin{itemize}
        \item $e \in S_3: e\of{1} = 1, e\of{2} = 2, e\of{3} = 3$.
        \item $\beta \in S_3$ where $\beta\of{1} = 2, \beta\of{2} = 3, \beta\of{3} = 1$.
    \end{itemize}
    Let's rewrite $\beta$ as follows: $\begin{bmatrix} 1 & 2 & 3 \\ 2 & 3 & 1 \end{bmatrix} = \R$.

    In general for any $\alpha \in S_3$, we may rewrite it as $\begin{bmatrix} 1 & 2 & 3 \\ \alpha\of{1} & \alpha\of{2} & \alpha\of{3} \end{bmatrix} = \R$.

    The number of permutations is given by the number of choices. This is $3! = 3 \cdot 2 \cdot 1$. We just proved that $\ord{S_3} = 3! = 6$.
    
    Similar reasoning tells us that $\ord{S_n} = n!$ for every $n \geq 1$.
\end{example}

\begin{question}
    Paul Mezo said we "know everything" about linear algebra. What does that mean?
\end{question}
\begin{answer}
    There are no unsolved problems in finite linear algebra.
\end{answer}

\section{Cycle Notation}

Consider $S_3$ and $\begin{bmatrix} 1 & 2 & 3 \\ 2 & 3 & 1 \end{bmatrix} \in S_3$. We rewrite this permutation as follows: $\cyc{1,2,3}$.

Notice that $\alpha = \cyc{1,2,3} \neq \cyc{1,3,2} = \beta$, but they are both 3-cycles.

Also, $\gamma = \cyc{1,2}$ is the permutation such that $\gamma\of{1} = 2, \gamma\of{2} = 1, \gamma\of{3} = 3$. It's a 2-cycle.

We omit 1-cycles.

The six permutations in $S_3$ in cycle notation are $e, \cyc{12}, \cyc{13}, \cyc{23}, \cyc{123}, \cyc{132}$.

\begin{example}[cycles of $S_4$]
    Consider $S_4$. $\ord{S_4} = 24 = 4!$.
    \begin{itemize}
        \item $e$,
        \item $\cyc{1,2}, \cyc{1,3}, \cyc{1,4}, \cyc{2,3}, \cyc{2,4}, \cyc{3,4}$
        \item $\cyc{1,2,3}, \cyc{1,3,4}, \dots$
        \item $\cyc{1,2,3,4}, \cyc{1,2,4,3}, \dots$
        \item $\cyc{1,2}\cyc{3,4}, \cyc{1,3}\cyc{2,4}, \cyc{1,4}\cyc{2,3}$
    \end{itemize}
\end{example}

\begin{definition}[disjoint cycles]
    Two cycles $\cyc{a_1, a_2, \dots, a_m}, \cyc{b_1, b_2, \dots, b_k} \in S_n$ are \textbf{disjoint} if $a_j \neq b_l$ for any $j, l$. Their product can be written equally in either order.
\end{definition}

Composition of permutations is interpreted as products of cycles as follows:

\begin{example}[compositions in $S_7$]
    \begin{itemize}
        \item $\cyc{6,2,3}\cyc{1,2} = \cyc{1,3,6,2}$
        \item $\cyc{1,2}\cyc{3,4,7}\cyc{2,3} = \cyc{2,4,7,3,1}$
        \item $\cyc{1,3}\cyc{2,4,5,6,7}\cyc{3,2}\cyc{1,2,5} = \cyc{2,6,7}\cyc{5,3,4}$
    \end{itemize}
\end{example}

\begin{remark}
    \begin{enumerate}
        \item Some authors move from left to right, one cycle to the next. We move from right to left.
        \item Cycles don't tell us which $S_n$ they live in.
    \end{enumerate}
\end{remark}

\begin{example}[powers of a $k$-cycle]
     Consider $\cyc{a_1,\dots,a_k} \in S_n$.
     \begin{enumerate}
        \item $\cyc{a_1,\dots,a_k}^2 = \cyc{a_1,a_2,a_3,\dots,a_k}\cyc{a_1,a_2,a_3,\dots,a_k}$ sends $a_1$ to $a_3$, and $a_l$ to $a_{l+2}$ if $l \leq k - 2$. Sends $a_{k-1} \to a_1$, $a_{k} \to a_2$.
        \item $\cyc{a_1,\dots,a_k}^j$ sends $a_l$ to $a_{(l + j) \mod k}$.
     \end{enumerate}
     In particular $\cyc{a_1,\dots,a_k}^k$ sends $a_l$ to $a_{(l+j) \mod k} = a_{l \mod k} = a_l$ for $1 \leq l \leq k$, so $\cyc{a_1,\dots,a_k}^k = e \implies \ord{\cyc{a_1,\dots,a_k}} = k$.
\end{example}

\begin{theorem}
    Every permutation in $S_n$ is a product of disjoint cycles. The products of disjoint cycles $\alpha, \beta \in S_n$ commute, i.e., $\alpha\beta = \beta\alpha$.
\end{theorem}
\begin{proof}
    Proof omitted.
\end{proof}

\begin{remark}
    Products of disjoint cycles can be written in more than one way to represent a single permutation in $S_n$.
    \[\cyc{1,2,3}\cyc{5,6} = \cyc{5,6}\cyc{1,2,3} = \cyc{5,6}\cyc{2,3,1}\]
\end{remark}

\begin{question}
    Dr. Mezo said they were unique "modulo" changing the order. Why use this language? What's the connection to modulo here?
\end{question}

\begin{definition}[the least common multiple]
    The \textbf{least common multiple} of $m,n \geq 1$ is the smallest positive integer $k$ such that $m$ divides $k$ and $n$ divides $k$. We write $k = \lcm\of{m,n}$.
\end{definition}

\begin{example}[finding LCM]
    \spacebeforelist
    \begin{enumerate}
        \item $\lcm\of{2,3} = 6$
        \item $\lcm\of{6,12} = 12$
        \item $\lcm\of{12,8} = \lcm\of{2^3 \cdot 3^1, 2^3 \cdot 3^0} = 2^3 \cdot 3^1 = 24$
    \end{enumerate}
\end{example}

\begin{theorem}
    Let $\alpha_1, \dots, \alpha_k \in S_n$ be disjoint cycles. Then $\ord{\alpha_1\dots\alpha_k} = \lcm\of{\ord{\alpha_1}, \dots, \ord{\alpha_k}}$
\end{theorem}
\begin{proof}
    Proof omitted.
\end{proof}

\begin{example}[the theorem]
    $\ord{(15)(37124)(986)} = 12 = \lcm\of{2,4,3}$
\end{example}

\begin{definition}[a transposition]
    A \textbf{transposition} in $S_n$ is a 2-cycle.
\end{definition}

\begin{example}[transpositions]
    Note
    \begin{itemize}
        \item $\cyc{1,2} = \cyc{2,1} = \cyc{1,2}\inv$
        \item $\cyc{1,2}\cyc{1,2} = \cyc{1}\cyc{2} = e$
        \item Similarly, $\cyc{a,b} = \cyc{b,a} = \cyc{ab}\inv$
        \item $\cyc{1,2,3} = \cyc{1,3}\cyc{1,2}$
        \item $\cyc{1,2,3} = \cyc{a,c}\cyc{a,b}$
    \end{itemize}
\end{example}

\begin{theorem}
    Every permutation in $S_n$ is a product of transpositions.
\end{theorem}
\begin{proof}
    $e = \cyc{1,2}\cyc{2,1} = \cyc{1,2}\cyc{1,2}$.

    Suppose $\sigma \in S_n, \sigma \neq e$. Then by a previous theorem, $\sigma = \beta_1 \dots \beta_k$ for disjoint cycles $\beta_1, \dots, \beta_k \in S_n$. If each $B_j$ is a product of transpositions then so is $\sigma$.
    
    Let $\beta = \cyc{a_1, \dots, a_k}$ be a $k$-cycle in $S_n$. Let's prove by induction on $k \geq 2$ that $\cyc{a_1, \dots, a_k} = \cyc{a_1 a_k} \cyc{a_1 a_{k-1}} \dots \cyc{a_1 a_2}$.
    
    Base case is obvious. Assume it's true for $k$.
    
    Let $\beta = \cyc{a-1, \dots, a_{k+1}}, \alpha = \cyc{a_1, a_{k+1}}, \gamma = \cyc{a_1, \dots a_k}$.

    By induction $\gamma = \cyc{a_1, \dots, a_k} = \cyc{a_1, a_k}\cyc{a_1,a_{k-1}}\dots\cyc{a_1,a_2}$.

    It suffices to show that $\beta = \alpha \gamma$.
    
    Let $1 \leq l \leq k - 1$. Then $\beta\of{a_l} = a_{l+1}$, and $\alpha\gamma\of{a_l} = \alpha\of{a_{l+1}}=a_{l+1} \implies \beta\of{a_l} = \alpha\gamma\of{a_l}$.

    So $\beta\of{a_k} = a_{k+1}$ and $\alpha\gamma\of{a_k} = \alpha\of{a_1} = a_{k+1} \implies \beta\of{a_k} = \alpha\gamma\of{a_k}$.

    So $\beta\of{a_{k+1}} = a_1$ and $\alpha\gamma\of{a_{k+1}} = \alpha\of{a_{k+1}} = a_1 \implies \beta\of{a_{k+1}} = \alpha\gamma\of{a_{k+1}}$

    Rest of the proof was erased before I could get to it :(
\end{proof}

\begin{lemma}
    If $e = \alpha_1 \dots \alpha_k$ is a product of transpositions $\alpha_1, \dots, \alpha_k \in S_n$, then $k$ is even.
\end{lemma}
\begin{proof}
    Proof omitted.
\end{proof}

\begin{theorem}
    Suppose $\alpha \in S_n$ and $\beta_1 \dots \beta_r = \alpha = \gamma \dots \gamma_s$ where $\beta_1, \dots, \beta_r, \gamma_1, \dots, \gamma_s \in S_n$ are transpositions.

    Then either $r$ and $s$ are both even, or they are both odd (i.e., $r \mod 2 = s \mod 2$).
\end{theorem}
\begin{proof}
    $\gamma_1 \dots \gamma_s = \beta_1 \dots \beta_r \implies \gamma_1\inv \gamma_1 \dots \gamma_s = \gamma_1\inv\beta_1\dots \beta_r \implies e = \gamma_s \dots \gamma_1 \beta_1 \dots \beta_r$, so the identity is a product of transpositions.

    By the lemma, $r + s$ is even.
\end{proof}

\begin{definition}[parity]
    We say that $\alpha \in S_n$ is \textbf{even} if it is a product of even number of transpositions, we say $\alpha$ is odd if it is a product of an odd number of transpositions.
\end{definition}

\begin{example}[parity of cycles]
    \spacebeforelist
    \begin{enumerate}
        \item $\cyc{a,b}$ odd
        \item $\cyc{a_1,a_2,a_3} = \cyc{a_1,a_3}\cyc{a_1,a_2}$ even
        \item $\cyc{a_1,a_2,a_3,a_4} = \cyc{a_1,a_4}\cyc{a_1,a_3}\cyc{a_1,a_2} = \cyc{a_1,a_4}\cyc{a_1,a_2,a_3}$
    \end{enumerate}
\end{example}

\begin{remark}
    A $k$-cycle is even for odd $k$ and is odd for even $k$.
\end{remark}

\begin{theorem}
    Let $A_n \subseteq S_n, n \geq 2$ be the subset of even elements in $S_n$. Then $A_n$ is a subgroup (called the \textbf{alternating group}).
\end{theorem}
\begin{proof}
    $e \in A_n$ so $A_n \neq \emptyset$. Suppose $\alpha = \beta_1 \dots \beta_s$ and $\sigma = \gamma_1 \dots \gamma_r$ for transpositions $\beta_1,\dots,\beta_s,\gamma_1,\dots,\gamma_r \in S_n$, i.e., $s$ and $r$ are even. Then $\alpha\sigma = \beta_1,\dots,\beta_s,\gamma_1,\dots,\gamma_r$ is a product of $r + s$ transpositions.

    Since $r + s$ is even, $\alpha \sigma \in A_n$.

    $\alpha\inv = \cyc{\beta_1 \dots \beta_s}\inv = \beta_s\inv \dots \beta_1\inv = \beta_s \dots \beta_1$ a product of $s$ transpositions. Since $s$ is even, $\alpha\inv \in A_n$.
\end{proof}

\begin{example}[alternating groups]
    \spacebeforelist
    \begin{itemize}
        \item $S_2 = \set{e, \cyc{1,2}} \supseteq \set{e} = A_2$
        \item $S_3 = \set{e} \cup$ 2-cycles $\cup$ 3-cycles $\supseteq$ $\set{e} \cup$ 3-cycles $ = \set{e, \cyc{1,2,3}, \cyc{1,3,2}} = A_3 = \cygr{\cyc{1,2,3}}$
        \item $S_4 = \set{e} \cup$ 2-cycles $\cup$ 3-cycles $\cup$ 4-cycles $\cup$ products of disjoint 2-cycles $\supseteq \set{e} \cup$ 3-cycles $\cup$ products of disjoint 2-cycles $ = A_4$
    \end{itemize}
\end{example}

\begin{theorem}
    $\ord{A_n} = \frac{n!}{2}$ for all $n \geq 2$.
\end{theorem}
\begin{proof}
    Recall that every element in $S_n$ is either even or odd. So $S_n = A_n \cup B$ is a disjoint union where $B$ is the set of odd permutations. So $\ord{S_n} = \ord{A_n} + \ord{B} \implies n! = \ord{A_n} + \ord{B}$.

    So if we prove $\ord{A_n} = \ord{B}$ then $n! = 2\ord{A_n} \implies \ord{A_n} = \frac{n!}{2}$.

    To prove $\ord{A_n} = \ord{B}$, we define $f: A_n \to B$ by $f\of{\alpha} = \cyc{1,2}\alpha$ for all $\alpha \in A_n$.

    Clearly, $\cyc{1,2}\alpha$ is odd since $\alpha$ is even. To show injectivity, suppose $\alpha, \beta \in A_n$ with $f\of{\alpha} = f\of{\beta} \implies \cyc{1,2}\alpha = \cyc{1,2}\beta \implies \cyc{1,2}\cyc{1,2}\alpha = \cyc{1,2}\cyc{1,2}\beta \implies \alpha = \beta$, so $f$ is injective.

    Suppose $\sigma \in B$. Then $f\of{\cyc{1,2}\sigma} = \cyc{1,2}\cyc{1,2}\sigma = \sigma$. This proves that $f$ is a bijection and $\ord{A_n} = \ord{B}$.
\end{proof}