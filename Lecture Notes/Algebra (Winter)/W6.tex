% camiregu 2024-feb-13
\chapter{}

%----------------------------------------------------------------------------------------
% DAY 1
\begin{theorem}
    Suppose $G_1 \oplus \cdots \oplus G_n$ is a direct product of finite groups and $(g_1, \dots, g_n) \in G_1 \oplus \cdots \oplus G_n$. Then $\ord{(g_1,\dots,g_n)} = \lcm\of{\ord{g_1}, \dots, \ord{g_n}}$.
\end{theorem}
\begin{proof}
    Let $t = \ord{(g_1, \dots, g_n)}$ and $m = \lcm\of{\ord{g_1},\dots, \ord{g_n}}$.

    Then $m = q_j \ord{g_j}$ for some $q_j \in \Z$ and any $1 \leq j \leq n$. So \[(g_1, \dots, g_n)^m = (g_1^m, \dots, g_n^m) = \left(g_1^{q_1 \ord{g_1}}, \dots, g_n^{q_n \ord{g_n}}\right) \]\[= \left(\left(g_1^{\ord{g_1}}\right)^{q_1}, \dots, \left(g_n^{\ord{g_n}}\right)^{q_n}\right) = (e^{q_1}, \dots, e^{q_n}) = (e, \dots, e),\]
    so $m$ is divisible by $\ord{(g_1, \dots, g_n)} = t \implies m \geq t$.

    In addition, \begin{align*}
        (e, \dots, e) &= (g_1, \dots g_n)^t \\
        &= (g_1^t, \dots, g_n^t) \\
        &\implies g_j^t = e \text{ for all } 1 \leq j \leq n \\
        &\implies \ord{g_j} \text{ divides } t \text{ for all } 1 \leq j \leq n \\
        &\implies t \text{ is a common multiple of } \ord{g_1}, \dots, \ord{g_n} \\
        &\implies t \geq m = \text{ the least common mutliple}.
    \end{align*}
\end{proof}

\begin{remark}
    For $n = p_1^{a_1} \dots p_k^{a_k}$, $m = p_1^{b_1} \dots p_k^{b_k}$, we have
    \begin{align*}
        \lcm\of{n,m} &= p_1^{\max\set{a_1,b_1}} \dots p_k^{\max\set{a_k,b_k}} \\
        \gcd\of{n,m} &= p_1^{\min\set{a_1,b_1}} \dots p_k^{\min\set{a_k,b_k}},
    \end{align*}
    so \[\lcm\of{n,m} \gcd\of{n,m} = p_1^{a_1 + b_1} \dots p_k^{a^k + b^k}\]
\end{remark}

\begin{example}[the direct product of prime groups]
    Suppose $\gcd{n,m} = 1$ are prime.

    $(1,1) \in \Z_n \oplus \Z_m$, $\ord{(1,1)} = \lcm\of{\ord{1}, \ord{1}} = \lcm\of{n,m} = (n)(m) = nm$.

    Also $\cygr{(1,1)} \subseteq \Z_n \oplus \Z_m$ and $\ord{\cygr{(1,1)}} = \ord{(1,1)} = nm \implies \cygr{(1,1)} = \Z_n \oplus \Z_m$.
\end{example}

\begin{example}[non-cyclic direct products]
    Take $n \geq 2$. Consider $\Z_n \oplus \Z_n$. Then $\lcm\of{n,n} = \gcd\of{n,n} = n$.

    Let $(a,b) \in \Z_n \oplus \Z_n$. By Lagrange's Theorem, $\ord{a}$ divides $n = \ord{\Z_n}$, so $\ord{b}$ divides $n$.
    Thus $n$ is a common multiple of $\ord{a}, \ord{b}$.

    By the theorem, $\ord{a,b} = \lcm\of{\ord{a}, \ord{b}} \leq n$. More directly $(a,b)^n = (a^n, b^n) = (e,e)$ by Lagrange's theorem since $n = \Z_n$. So $\ord{(a,b)}$ divides $n$.

    It follows $\ord{\cygr{(a,b)}} = \ord{(a,b)} \leq n < n^2 = \ord{\Z_n \oplus \Z_n}$.

    Therefore $\cygr{(a,b)} \neq \Z_n \oplus \Z_n$ and $\Z_n \oplus \Z_n$ is not cyclic.
\end{example}

\begin{theorem}
    Suppose $G$ and $H$ are finite cyclic groups ($G \cong \Z_n, H \cong Z_m$). Then $G \oplus H$ is cyclic if and only if $1 = \gcd\of{\ord{G}, \ord{H}}$.
\end{theorem}
\begin{proof}
    Let $n = \ord{G}$ and $m = \ord{H}$. Suppose $G = \cygr{g}$ and $H = \cygr{h}$ so that $\ord{g} = n$ and $\ord{h} = m$.

    "$\impliedby$" Suppose $\gcd\of{n,m} = 1$. Then $nm = \lcm\of{n,m} \gcd\of{n,m} = \lcm\of{n,m}$ by theorem 8.1 $\ord{(g,h)} = \lcm\of{n,m} = nm$.

    Moreover $\cygr{(g,h)} \subseteq G \oplus H$ and $\ord{\cygr{(g,h)}} = nm = \ord{G} \oplus \ord{H}$. So $\cygr{(g,h)} = G \oplus H$ and $G \oplus H$ is cyclic.

    "$\implies$" Suppose $G \oplus H = \cygr{(a,b)}$ so that $\ord{(a,b)} = \ord{\cygr{(a,b)}} = \ord{G \oplus H} = nm$.

    Let $d = \gcd\of{n,m}$. Then $(a,b)^{\frac{nm}{d}} = \left(a^{\frac{nm}{d}}, b^{\frac{nm}{d}}\right) = \left(\left(a^n\right)^{\frac{m}{d}}, \left(b^m\right)^{\frac{n}{d}}\right) = \left( e^{\frac{m}{d}}, e^{\frac{n}{d}}\right) = (e,e)$.

    Thus $\ord{(a,b)}$ divides $\frac{nm}{d}$ implies $nm$ divides $\frac{nm}{d} \implies d = 1$
\end{proof}

\begin{example}[a non-cyclic group]
    $\Z_2 \oplus \Z_4$ is not cyclic since $\gcd (2,4) = 2 \neq 1$.
\end{example}

\begin{example}[$\Z_2 \oplus \Z_{10} \oplus \Z_{13}$]
    $\ord{(1,1,1)} = \lcm\of{\ord{1},\ord{1},\ord{1}} = \lcm\of{2,10,13} = (2)(5)(13) = 130 \neq 260 = \ord{\Z_2 \oplus \Z_{10} \oplus \Z_{13}}$.

    $\cygr{(1,1,1) \subseteq \Z_2 \oplus \Z_{10} \oplus \Z_13}$ and $\ord{\cygr{(1,1,1)}} = \ord{(1,1,1)} \neq \ord{\Z_2 \oplus \Z_{10} \oplus \Z_{13}}$. That is, the element $(1,1,1)$, made up of the generators of each group, does not generate the direct product of the groups. Observe that $\gcd\of{2,10,13} = 1$.
\end{example}

\begin{corollary}
    Suppose $G_1, \dots, G_n$ are finite cyclic groups. Then $G_1 \oplus \cdots \oplus G_n$ is cyclic if and only if $\gcd\of{\ord{G_j}, \ord{G_k}} = 1$ for all $j \neq k$.
\end{corollary}
\begin{proof}
    Proof omitted.
\end{proof}

\begin{corollary}
    Let $m = n_1 \dots n_k$. Then \[\Z_m \cong \Z_{n_1} \oplus \cdots \Z_{n_k} \iff \gcd\of{n_j, n_l} = 1 \forall j \neq l.\]
\end{corollary}
\begin{proof}
    (Sketch). Take $G_j = \Z_{n_j}$. Then $\ord{G_1 \oplus \cdots \oplus G_k} = n_1 n_2 \dots n_k = m$. By corollary 6.3 $G_1 \oplus \cdots \oplus G_k \cong Z_m \iff \gcd\of{n_j, n_l} = 1 \forall j \neq l$.
\end{proof}

\begin{example}[using the theorem]
    Proofs left as an exercise, but notice $12 = 2^2 \cdot 3 = 6 \cdot 2 = 4 \cdot 3$, but by corollary 6.4,
    \begin{itemize}
        \item $\Z_{12} \not\cong \Z_2 \oplus \Z_2 \oplus \Z_3$
        \item $\Z_{12} \not\cong \Z_6 \oplus \Z_3$
        \item $\Z_{12} \cong \Z_4 \oplus \Z_3$
    \end{itemize}
\end{example}

It is not hard to see that if $m$ has prime factorization $m = p_1^{a_1} \dots p_k^{a_k}$ ($a_j \geq 0$), then $\Z_m \cong \Z_{p_1^{a_1}} \oplus \cdots \oplus \Z_{p_p^{a_k}}$ by corollary 6.4 (since their $\gcd$ is 1).

%----------------------------------------------------------------------------------------
% DAY 2