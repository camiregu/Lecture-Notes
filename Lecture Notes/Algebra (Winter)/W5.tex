% camiregu 2024-feb-6
\chapter{Cosets and Lagrange's Theorem}

%----------------------------------------------------------------------------------------
% DAY 1

\begin{definition}[a coset]
    A (left) \textbf{coset of H} is of the form $xH = \set{xh: h \in H}$ where $x \in G$.
    A (right) \textbf{coset of H} is of the form $Hx = \set{hx: h \in H}$ where $x \in G$.

    In both cases, $x$ is called a (coset) \textbf{representative for $xH$} (or $Hx$). $\ord{xH}$ is the number of elements in $xH$.
\end{definition}

\begin{lemma}
    Let $H$ be a subgroup of $G$ and $x,y \in G$. Then 
    \begin{enumerate}
        \item $x \in xH$
        \item $xH = H \iff x \in H$
        \item $x(yH) = (xy)H$
        \item $xH = yH \iff x \in yH$
        \item Either $xH = yH$ or $xH \cap yH = \emptyset$
        \item $xH = yH \iff y\inv x \in H \iff x\inv y \in H$
        \item $\ord{xH} = \ord{H}$
        \item $xH = Hx \iff xHx\inv = H$
        \item $xH$ is a subgroup $\iff$ $x \in H \iff xH = H$   
    \end{enumerate}
\end{lemma}

%----------------------------------------------------------------------------------------
% DAY 2

\begin{theorem}[Lagrange's Theorem]
    Suppose $H$ is a subgroup of a finite group $G$. Then $\ord{H}$ divides $\ord{G}$ and the number of cosets is $\frac{\ord{G}}{\ord{H}}$.
\end{theorem}
\begin{proof}
    $G = x_1 H \bigcup \cdots \bigcup x_m H$ a disjoint union $\implies$ $\ord{G} = \sum_{j=1}^m \ord{x_j H} = m \ord{H}$.
\end{proof}

\begin{definition}[coset spaces and index]
    Suppose $H$ is a subgroup of $G$, then the number of (left) cosets is called the \textbf{index of H in G} and is denoted by $\ord{G:H}$. The set of (left) cosets is denoted by $\cosets{G}{H} = \set{gH: g \in G}$ and is called the \textbf{coset space}. So \[\ord{\cosets{G}{H}} = \ord{G:H}\]
\end{definition}

\begin{example}
    If $G$ is finite then $\ord{\cosets{G}{H}} = \ord{G:H} = \frac{\ord{G}}{\ord{H}}$.
    \begin{itemize}
        \item Take $G = \Z$ and $H = \cygr{2}$. Then $\cosets{\Z}{\cygr{2}} = \set{0 + \cygr{2}, 1+ \cygr{2}} \implies \ord{\Z:\cygr{2}} = 2$.
        \item $\ord{\Z :\cygr{3}} = 3$
        \item $\ord{\Z :\cygr{0}} = \infty$
        \item $\ord{D_3 : \cygr{\rho}} = \frac{6}{3} = 2$ by Lagrange's theorem.
    \end{itemize}
\end{example}

\begin{corollary}
    Suppose $G$ is finite and $x \in G$. Then $\ord{x}$ divides $\ord{G}$.
\end{corollary}
\begin{proof}
    $\ord{x} = \ord{\cygr{x}}$ divides $\ord{G}$ by Lagrange.
\end{proof}

\begin{corollary}
    Suppose $\ord{G} = p$ is a prime number. Then $G$ is cyclic and $G \cong \Z_p$.
\end{corollary}
\begin{proof}
    Suppose $x \in G, x \neq e$. Then $1 \neq \ord{\cygr{x}}$ divides $p$ by Lagrange's theorem. So $\ord{\cygr{x}} = p$. However $\cygr{x} \subseteq G$ so $\cygr{x} = G$.

    For the desired isomorphism let $\phi\of{k} = x^k$, $0 \leq k \leq p-1$.
\end{proof}

\begin{corollary}
    Suppose $G$ is finite and $x \in G$. Then $x^{\ord{G}} = e$.
\end{corollary}

\begin{corollary}[Fermat's Little Theorem]
    Suppose $m \in \Z$ and $p$ is prime. Then $m^p \mod p = m \mod p$.
\end{corollary}

\section{External Direct Products}

\begin{definition}[a direct product]
    Suppose $G_1, \dots, G_n$ are groups. Then $G_1 \oplus \cdots \oplus G_n = G_1 \times \cdots \times G_n = \set{(g_1, \dots, g_n): g_i \in G_i}$ together with multiplication defined by $(g_1,\dots,g_n)(g_1',\dots,g_n')$. $\ord{G_1 \oplus \cdots \oplus G_n} = \Pi_{i=1}^n \ord{G_i}$
\end{definition}

\begin{theorem}
    Suppose $G_1 \oplus \cdots \oplus G_n$ is a direct product of groups and $(g_1, \dots, g_n) \in G_1 \oplus \cdots \oplus G_n$. Then $\ord{(g_1, \dots, g_n)} = \lcm(\ord{g_1}, \dots, \ord{g_n})$.
\end{theorem}