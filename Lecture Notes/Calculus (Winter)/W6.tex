% camiregu 2024-feb-6
\chapter{The integral over a bounded set}

%----------------------------------------------------------------------------------------
% DAY 1
Let $f: \Omega \to \R$ be a bouded function where $\Omega \subseteq \R^n$ is a bounded set. Thus $\Omega$ is contained in a rectangle $R$. We can try to define \[\int_\Omega f = \int_R \tilde{f}\] where $\tilde{f}$ is given by $\tilde{f}\of{x} = \left\{\begin{matrix} f\of{x} & \text{when } x \in R \\ 0 & \text{otherwise} \end{matrix}\right.$.

\begin{definition}[the characteristic function of a set]
    Let $\Omega \subseteq \R^n$. The characteristic (indicator) function $\chi_\Omega$ of $\Omega$ is defined by $\chi_\Omega\of{x} = \left\{ \begin{matrix} 1 & \text{when }x \in \Omega \\ 0 & \text{otherwise} \end{matrix} \right.$.
\end{definition}

Clearly, given a function $f: S \to \R$ where $\Omega \subseteq S$, $\tilde{f} = f \cdot \chi_\Omega$ is a function with $\tilde{f}\of{x} = f\of{x}$ for all $x \in \Omega$ and $\tilde{f}\of{x} = 0$ for all $x \in S \backslash \Omega$. We will abuse this notation a bit and consider $f \cdot \chi_\Omega$ as the notation for the function $\tilde{f}: \R^n \to \R$ such that $\tilde{f}\of{x} = f\of{x}$ for all $x \in \Omega$ and $\tilde{f}\of{x} = 0$ for all $x \in \R^n \backslash \Omega = \Omega^C$.

Our proposed definition of $\int_\Omega f$ is then \[\int_\Omega = \int_R f \cdot \chi_\Omega \text{  ($R \geq \Omega$)}\]

\begin{lemma}
    Let $R$ and $S$ be rectangles in $\R^n$ with $S \subseteq R$ and let $g: R \to \R$ be a bounded function with $g\of{x} = 0$ for all $x \in R \backslash S$. If $g$ is integrable over $S$, then $g$ is integrable over $R$ and $\int_R g = \int_S g$.
\end{lemma}

\begin{proposition}
    Let $f: \Omega \to \R$ be a bounded function where $\Omega \subseteq \R^n$ is bounded. If $R_1$ and $R_2$ are rectangles with $\Omega \subseteq R_1$ and $\Omega \subseteq R_2$, then $f \cdot \chi_\Omega$ is integrable over $R_1$ if and only if it is integrable over $R_2$ and $\int_{R_1} f \cdot \chi_\Omega = \int_{R_2} f \cdot \chi_\Omega$
\end{proposition}
\begin{proof}
    Clearly, $\Omega \subseteq R_1 \cap R_2$.
    \begin{itemize}
        \item Case 1: $R_1$ and $R_2$ intersect along their boundaries. Then $R_1 \cap R_2$ is not a rectangle and has zero $n$-dimensional volume. But $\Omega \subseteq R_1 \cap R_2$ also has zero $n$-dimensional volume. Hence $\set{x \in R_1: (f \cdot \chi_\Omega)\of{x} \neq 0},\set{x \in R_2: (f \cdot \chi_\Omega)\of{x} \neq 0} \subseteq \Omega$ must also have zero $n$-dimensional volume.
        
        Hence, $f \cdot \chi_\Omega$ is integrable over both $R_1$ and $R_2$ and $\int_{R_1} f \cdot \chi_\Omega = 0 = \int_{R_2} f \cdot \chi_\Omega$.
        \item Case 2: $R_1 \cap R_2$ is a rectangle. Suppose $f: \chi_\Omega$ is integrable over $R_1$. Then (as $R_1 \cap R_2 \subseteq R_1$) is integrable over $R_1 \cap R_2$ (by an earlier result) and as $(f \cdot \chi_\Omega)\of{x} = 0$ for all $x \in R_1 \backslash (R_1 \cap R_2)$, the last lemma yields $\int_{R_1} f \cdot \chi_\Omega =  \int_{R_1 \cap R_2} f \cdot \chi_\Omega$, the lemma shows that $f \cdot \chi_\Omega$ is integrable over $R_2$ and $\int_R f \cdot \chi_\Omega = \int_{R_1 \cap R_2} f \cdot \chi_\Omega = \int_{R_1} f \cdot \chi_\Omega$.
    \end{itemize}
    The proof of the converse is analogous
\end{proof}

\begin{definition}[the integral over a bounded set]
    Let $\Omega \subseteq \R^n$ be a bounded set and $f: \Omega \to \R$ a bounded function. We show that $f$ is integrable over $\Omega$ if the function $f \cdot \chi_\Omega$ is integrable over (any) rectangle $R \supseteq \Omega$.

    Then the integral of $f$ over $\Omega$ is defined by \[\int_\Omega f = \int_R f \cdot \chi_\Omega.\]
\end{definition}

Notice that if $\Omega$ is a rectangle, then the new definition agrees with the old one.

\begin{theorem}
    The basic properties of integrals established in theorems 1*, 2*, and corollary *3 for integration over rectangles remain true for integration over bounded sets.
\end{theorem}

\begin{corollary}[(to the new Theorem *2)]
    Let $f: \Omega \to \R$ be a bounded function where $\Omega \subseteq \R^n$ has zero volume. Then $f$ is integrable over $\Omega$ and $\int_\Omega f = 0$.
\end{corollary}
\begin{proof}
    $N = \set{x \in \Omega: f\of{x} \neq 0} \subseteq \Omega$ so it has zero volume.
\end{proof}

\begin{theorem}
    Let $\Omega_1, \Omega_2 \subseteq \R^n$ be bounded where $\Omega_1 \subseteq \Omega_2$ and let $f: \Omega_2 \to [0, \infty]$ be integrable over both $\Omega_1$ and $\Omega_2$. Then \[\int_{\Omega_1} f \leq \int_{\Omega_2} f.\]
\end{theorem}
\begin{proof}
    If $R$ is a rectangle with $\Omega_2 \subseteq R$, then $\Omega_1 \subseteq R$ and $f \cdot \chi_{\Omega_1} \leq f \cdot \chi_{\Omega_2}$. So $\int_{\Omega_1} f = \int_R f \cdot \chi_{\Omega_1} \leq \int_R f \cdot \chi_{\Omega_2} = \int_{\Omega_2} f$.
\end{proof}

\begin{lemma}
    If $\Omega_1, \Omega_2 \subseteq \R^n$ then 
    \begin{enumerate}
        \item $\chi_{\Omega_1 \cap \Omega_2} = \chi_{\Omega_1} \cdot \chi_{\Omega_2} = \frac{1}{2} (\chi_{\Omega_1} + \chi_{\Omega_2} - \abs{\chi_{\Omega_1} - \chi_{\Omega_2}})$
        \item $\chi_{\Omega_1 \cup \Omega_2} = \chi_{\Omega_1} + \chi_{\Omega_2} - \chi_{\Omega_1 \cap \Omega_2}$
        \item $\chi_{\Omega_1 \backslash \Omega_2} = \chi_{\Omega_1} \backslash \chi_{\Omega_1 \cap \Omega_2}$
    \end{enumerate}
\end{lemma}

\begin{theorem}
    Let $\Omega_1, \Omega_2 \subseteq \R^n$ be bounded and let $f: \Omega_1 \cup \Omega_2 \to \R$ be integrable over $\Omega_1$ and $\Omega_2$. Then $f$ is integrable over $\Omega_1 \cup \Omega_2$, $\Omega_1 \cap \Omega_2$, and $\Omega_1 \backslash \Omega_2$ and
    \[\int_{\Omega_1 \cup \Omega_2} f = \int_{\Omega_1} f + \int_{\Omega_2} f - \int_{\Omega_1 \cap \Omega_2} f,\]
    \[\int_{\Omega_1 \backslash \Omega_2} f = \int_{\Omega_1} f - \int_{\Omega_1 \cap \Omega_2} f.\]
\end{theorem}
\begin{proof}
    First assume that $f \geq 0$. Then by the lemma \[f \cdot \chi_{\Omega_1 \cap \Omega_2} = \frac{1}{2} \left(f \cdot \chi_{\Omega_1} + f \cdot \chi_{\Omega_2} - \abs{f \cdot \chi_{\Omega_1} - f \cdot \chi_{\Omega_2}}\right). (*)\]
    If $R$ is a rectangle with $\Omega_1 \cup \Omega_2 \subseteq R$, then $f \cdot \chi_{\Omega_1}$ and $f \cdot \chi_{\Omega_2}$ are integrable over $R$. Using (*), $f \cdot \chi_{\Omega_1 \cap \Omega_2}$ is integrable over $R$. So $f$ is integrable over $\Omega_1 \cap \Omega_2$.

    To obtain the same conclusion for general $f$, note that $f = f_+ - f_-$ where $f_+ = \frac{1}{2}(f + \abs{f}), f_- = \frac{1}{2}(\abs{f} - f)$ are non-negative. Note that $f_+$ and $f_-$ are integrable over $\Omega_1$ and $\Omega_2$. As $f_{\pm} \geq 0$, then from the 1st part of the proof $f = f_+ - f_-$ is integrable over $\Omega_1 \cap \Omega_2$.

    The remaining is a statement of formulas:
    \[f \cdot \chi_{\Omega_1 \cup \Omega_2} = f \cdot \chi_{\Omega_1} + f \cdot \chi_{\Omega_2} - f \cdot \chi_{\Omega_1 \cap \Omega_2}, \text{ and}\]
    \[f \cdot \chi_{\Omega_1 \backslash \Omega_2} = f \cdot \chi_{\Omega_1} - f \cdot \chi_{\Omega_1 \cap \Omega_2}\]
    and integration over a rectangle $R \supseteq \Omega_1 \cup \Omega_2$.
\end{proof}

Recall that a continuous function on a rectangle $R$ is always integrable over $R$. We also know that if $f$ is integrable over and $S \subseteq R$ is a subrectangle then $f$ is integrable over $S$; however, this is not always true with rectangles replaced by bounded sets.

\begin{example}[a non-integrable bounded set]
    Let $\Omega = ([0,1] \times [0,1]) \cap (\Q \times \Q)$. Then $\chi_\Omega$ is not integrable over $R = [0,1] \times [0,1]$. Thus the constant function $f\of{x} = 1$ for all $x \in R$ is not integrable over $\Omega$ (as $f \cdot \chi_\Omega = \chi_\Omega$), but $f$ is trivially integrable over $R$.
\end{example}

\begin{lemma}
    If $\Omega \subseteq \R^n$, then $D = \set{x \in \R^n: \chi_\Omega \text{ is discontinuous at } x} = \partial\Omega$
\end{lemma}

%----------------------------------------------------------------------------------------
% DAY 2